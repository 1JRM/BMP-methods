%\documentclass[twocolumn]{article}
\documentclass[]{article}
\usepackage[version=3]{mhchem}
\usepackage{sectsty}
\usepackage{amsmath}
\usepackage[flushleft]{threeparttablex} % For table notes
\usepackage{rotating} 
\usepackage{longtable}
\usepackage{hyperref}
% For decimal alignment in tables
\usepackage{dcolumn}

% For dcolumn
\newcolumntype{.}{D{.}{.}{3}}

\sectionfont{\large}

\newcommand{\unit}[1]{\ensuremath{\, \mathrm{#1}}}

\title {Calculation of biochemical methane potential (BMP)}
\author{Sasha D. Hafner\\
\\
\texttt{sasha@hafnerconsulting.com} (S. D. Hafner)
} 

\begin{document}
\maketitle

\section{BMP-methods}
File version 1.0. 
This file is from the GitHub repository BMP-methods.
For more information, visit BMP-methods at \url{https://github.com/sashahafner/BMP-methods}.

\section{Description}
This document describes how to calculate biochemical methane potential (also called biomethane potential) from laboratory measurements.
The calculations are based on standardized \ce{CH4} volume produced in bottles with 1) inoculum only and 2) with substrate and inoculum, along with the quantity of inoculum and substrate added to each bottle.

\section{Selection of a BMP duration}
The time at which to evaluation BMP, i.e., the length of the incubation is not standardized among groups.
It may be fixed prior starting a BMP trials (e.g., 30 d) or may be calculated after (or as the trial is running) based on the rate of \ce{CH4} production.
Regardless, it is important that the time is identical for both inoculum-only and inoculum + substrate bottles when carrying out calculation of BMP.
This does not mean it cannot vary among substrates from within the same BMP trial.

\section{Calculation of BMP}

These calculations require the following variables.
Units may differ, but typical units are listed below.
\begin{itemize}
  \item $V_{CH_4, S, i}$, the standardized volume of \ce{CH4} produced in bottle $i$ with inoculum and substrate at time $t$ (mL)
  \item $V_{CH_4, I, j}$, the standardized volume of \ce{CH4} produced in bottle $j$ with inoculum only at time $t$
  \item $m_{I, i}$, the mass of inoculum (typically as-measured (fresh) mass) originally added to bottle $i$
  \item $m_{VS, S, i}$, the mass of substrate volatile solids (VS) originally added to bottle $i$
\end{itemize}

Productivity of inoculum is calculated separately for each inoculum-only bottle by Eq. (1).
\begin{equation}
  \label{eq:inoc_production}
  v_{CH_4, I, j} = V_{CH_4, I, j} / m_{I, j} 
\end{equation}
And from these, a mean value is calculated as 
\begin{equation}
  \label{eq:inoc_productivity}
  \bar{v}_{CH_4, I} = \sum_1 ^k v_{CH_4, I, i} / k
\end{equation}
where $k$ = the number of inoculum-only bottles.

Net \ce{CH4} production from inoculum + substrate bottles, i.e., an estimate of \ce{CH4} production derived from substrate only\footnote{This calculation is based on the assumption of additivity for \ce{CH4} production, i.e., production of \ce{CH4} from inoculum is not affected by the presence of substrate. This is almost certainly incorrect, but similar results even when varying the inoculum-to-substrate ratio suggest it is not a large source of error.} is calculated as given in Eq. (3).
\begin{equation}
  \label{eq:net_CH4}
  V_{CH_4, S, i, net} = V_{CH_4, S, i} - \bar{v}_{CH_4, I} \cdot m_{I, i}
\end{equation}
Note that the units on inoculum mass are completely irrelevant and have no effect of results, as long as they are sufficiently precise.
Fresh (wet) mass is recommended, although dry or VS mass could be used.\footnote{Any error in determination of inoculum dry matter or VS here is exactly canceled by the combination of Eqs. (1) and (3), so has no effect.}

Bottle yield is calculated by normalizing net \ce{CH4} production by substrate VS mass.
\begin{equation}
  \label{eq:yield}
  B_{i} = V_{CH_4, S, i, net} / m_{VS, S, i}
\end{equation}
Finally, BMP for a particular substrate is taken as the mean of these values.
\begin{equation}
  \label{eq:BMP}
  \bar{B} = \sum_1 ^n B_{i} / n
\end{equation}

\end{document}
